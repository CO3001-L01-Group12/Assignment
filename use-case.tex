\subsection{Draw a use-case diagram for the whole system.
Choose at least one important module and draw its use-case diagram, as 
well as describe the use-cases using a table format}

\begin{figure}[H]
  \centering
  \vspace{0.5 cm}
  \includegraphics [scale=0.35] {data/1_3all.png}
  \vspace{1 cm}
  \caption{1.3 All USE CASE DIAGRAM }
\end{figure}
\subsubsection{Quản lý máy in}
\begin{figure}[H]
    \centering
    \includegraphics[width = 1\textwidth]{Images/use-case-diagram/manage_printers.png}
    \newline
    \newline
    
    \caption{Use-case diagram của nhánh chức năng quản lý máy in của SPSO}
    \label{fig:enter-label}
\end{figure}
\subsubsubsection{Xem thông tin máy in}
\begin{xltabular}{\textwidth}{|c|X|}
    \hline
    \textbf{Tên use-case} & \textbf{Xem thông tin máy in} \\
    \hline
    Actor & SPSO \\
    \hline
    Descriptions & Các SPSO sử dụng chức năng này để xem thông tin máy in. \\
    \hline 
    Precondition &Có danh sách các máy in hiện có trong hệ thống. 
    \newline SPSO đang ở trang "Quản lý"
    \\
    \hline
    Postcondition &Hệ thống hiển thị thông tin của máy in .\\
    \hline
    Trigger &SPSO chọn vào nút "Quản lý" ở trên thanh điều hướng.\\
    \hline
    Normal Flows & 
    1. Hệ thống mở cửa sổ quản lý máy in.
    \newline
    2. Hệ thống lấy danh sách các máy in.
    \newline
    3. Hệ thống hiển thị danh sách các máy in lên màn hình.
    \newline
    4. SPSO chọn vào máy in muốn xem thông tin.
    \newline
    5. Hệ thống lấy thông tin chi tiết của máy in và hiển thị cho SPSO.
    \newline
    \\
    \hline
    Exception Flows & None
    \\
    \hline
    Alternative Flows &
    - Tìm kiếm máy in bằng id sau bước 3.
    \newline
    - \textit{Extended points}: Bật/tắt máy in (tại bước 5)
    \\
    \hline
\end{xltabular}
\subsubsubsection{Thêm máy in}
\begin{xltabular}{\textwidth}{|c|X|}
    \hline
    \textbf{Tên use-case} & \textbf{Thêm máy in} \\
    \hline
    Actor & SPSO \\
    \hline
    Descriptions & Các SPSO sử dụng chức năng này để thêm máy in mới. \\
    \hline 
    Precondition &SPSO đang ở trang "Quản lý". \\
    \hline
    Postcondition &Hệ thống cập nhật máy in mới vào database .\\
    \hline
    Trigger &SPSO chọn vào nút "Thêm máy in" ở trong giao diện "Quản lý".
    \\
    \hline
    Normal Flows & 
    1. Hệ thống hiển thị trang để SPSO nhập thông tin máy in.
    \newline
    2. SPSO nhập các thông tin của máy in và ấn vào nút "Thêm".
    \newline
    3. Hệ thống cập nhật thông tin của máy in vừa được thêm vào.
    \newline
    4. Hệ thống hiển thị thông báo thêm máy in thành công.
    \newline
    \\
    \hline
    Exception Flows & Ở bước 2:\newline
    2a. SPSO nhập vào thông tin của một máy in đã có trong hệ thống.\newline
    2b. Hệ thống sẽ hiển thị thông báo máy in này đã có trong hệ thống và yêu cầu SPSO nhập lại thông tin.
    \\
    \hline
    Alternative Flows & None
    
    \hline
\end{xltabular}
\subsubsubsection{Bật/tắt máy in}
\begin{xltabular}{\textwidth}{|c|X|}
    \hline
    \textbf{Tên use-case} & \textbf{Bật/tắt máy in} \\
    \hline
    Actor & SPSO \\
    \hline
    Descriptions & Các SPSO sử dụng chức năng này để bật/tắt các máy in đang được quản lý trong hệ thống. \\
    \hline 
    Precondition &Có danh sách các máy in hiện có trong hệ thống. 
    \newline
    SPSO đang ở trang "Quản lý"
    \\
    
    \hline
    Postcondition &Máy in được chọn hiển thị trạng thái "bật" hoặc "tắt" do SPSO chọn.\\
    \hline
    Trigger &SPSO chọn vào nút "Bật/tắt" ở trong phần thông tin của máy in ngay ở giao diện hiển thị danh sách các máy in hoặc ở trong giao diện hiển thị thông tin chi tiết của một máy in cụ thể.\\
    \hline
    Normal Flows & 
    1. SPSO chọn máy in muốn thao tác.
    \newline
    2. Hệ thống hiển thị thông tin của máy in đó.
    \newline
    3. SPSO chọn vào nút "bật/tắt" để chuyển đổi qua lại giữa hai trạng thái của máy in.
    \newline
    4. Hệ thống lưu lại trạng thái của máy in.
    \\
    \hline
    Exception Flows & None
    \\
    \hline
    Alternative Flows & None
    \\
    \hline
\end{xltabular}


\subsubsection{In tài liệu}
\begin{figure}[H]
    \centering
    \includegraphics[width = 1\textwidth]{Images/use-case-diagram/printing.png}
    \caption{Use-case diagram của nhánh chức năng in tài liệu của sinh viên}
    \label{fig:enter-label}
\end{figure}
\begin{xltabular}{\textwidth}{|c|X|}
    \hline
    \textbf{Tên use-case} & \textbf{In tài liệu} \\
    \hline
    Actor & Sinh viên \\
    \hline
    Descriptions & Sinh viên sử dụng chức năng này để in tài liệu. \\
    \hline 
    Precondition & Sinh viên muốn in tài liệu trước hết phải đăng nhập, tải tài liệu lên, chọn máy in và các thuộc tính in. \\
    \hline
    Postcondition & Sinh viên in thành công và trở về màn hình chính.\\
    \hline
    Normal Flows & 
    1. Nhấn chọn "In tài liệu"
    \newline
    2. Hệ thống sẽ chuyển hướng trang in tài liệu.
    \newline
    3. Tại giao diện này, sinh viên sẽ tiến hành thực hiện 3 bước : 
    \newline
    4. Sinh viện chọn tài liệu cần tải lên và nhấn nút "Tải tài liệu lên".
    \newline
    5. Sinh viên chọn loại máy in để thực hiện in.
    \newline
    6. Sinh viên chọn các thuộc tính như cỡ giấy, số trang cần in, format, ...để in.
    \newline
    7. Sinh viên bấm nút "In".
    \newline
    8. Hệ thống sẽ tiến hành in và thông báo kết quả cho sinh viên.
    \\
    \hline
    Exception Flows & 
    Ở bước 4:
    \newline
    4a. Sinh viên tải sai định dạng của tài liệu 
    \newline
    4b. Hệ thống thông báo lỗi và yêu cầu sinh viên tải lại tài liệu khác.
    \newline
    Ở bước 6:
    \newline
    6a. Sinh viên không đủ số lượng giấy hiện có trong tài khoản để in.
    \newline
    6b. Hệ thống thông báo lỗi và chuyển hướng qua BKPay để sinh viên thực hiện mua thêm giấy in. \\
    \hline
    Alternative Flows & 
    Ở bước 5 và bước 6, nếu sinh viên không thực hiện các bước này thì hệ thống sẽ tự động sử dụng các option mặc định và chuyển sang bước 7. \\
    \hline
\end{xltabular}

\subsubsection{Tải tài liệu}
\begin{figure}[H]
    \centering
    \includegraphics[width = 1\textwidth]{Images/use-case-diagram/uploadfile.png}|
    \vspace{0.1cm}
    \caption{Use-case diagram của nhánh chức năng tải tệp tài liệu lên hệ thống cho sinh viên}
    \label{fig:enter-label}
\end{figure}
\begin{xltabular}{\textwidth}{|c|X|}
    \hline
    \textbf{Tên use-case} & \textbf{Tải tài liệu} \\
    \hline
    Actor & Sinh viên \\
    \hline
    Descriptions & Sinh viên sử dụng chức năng này nhằm tải tệp tài liệu lên hệ thống để chuẩn bị in ấn. \\
    \hline 
    Precondition & Sinh viên muốn tải tệp tài liệu lên hệ thống trước hết phải xác thực và đăng nhập vào hệ thống. \\
    \hline
    Postcondition & Tệp tài liệu đã được tải lên hệ thống và sẵn sàng để in.\\
    \hline
    Normal Flows & 
    1. Nhấn chọn "Tải tài liệu."
    \newline
    2. Hệ thống hiển thị giao diện cho phép sinh viên chọn tệp tài liệu từ máy tính hoặc thiết bị di động của họ.
    \newline
    3. Sinh viên tìm và chọn tệp tài liệu mà họ muốn in.
     \newline
    4. Sinh viên nhấn chọn "Tải lên" để bắt đầu quá trình tải tệp lên.
    \newline
    5. Hệ thống xử lý và lưu trữ tệp tài liệu trên hệ thống.
    \newline
    6. Hệ thống gửi thông báo xác nhận thành công cho sinh viên và hiển thị tệp tải lên trong danh sách tài liệu của họ.\\
    \hline
    Exception Flows & 
    None.\\
    \hline
    Alternative Flows & None.\\
    \hline
\end{xltabular}
\newpage
