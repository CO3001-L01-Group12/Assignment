\section{TASK 3: ARCHITECTURE DESIGN}
\subsection{Architecture}
\subsubsection{Architecture design}

\begin{figure}[H]
    \centering
\includegraphics[scale=0.6]{Images/architecture-diagram/box-line.png}
    \vspace{0.1cm}
    \par
    \caption{Box-line diagram}
\end{figure}
\begin{itemize}
\item Hệ thống sử dụng kiến trúc lớp (Layered architecture) để thiết kế.
\item Presentation Layer (Lớp trình bày): Đây là lớp giao diện người dùng, chịu trách nhiệm hiển thị thông tin cho người dùng và nhận các yêu cầu từ họ.\par    
\begin{itemize}
\item Student Interface: Chế độ mặc định khi truy cập vào trang chính của hệ thống. Dành cho sinh viên và cho phép họ thực hiện việc in ấn tài liệu khi có nhu cầu. 
\item SPSO Interface: Chế độ dành cho SPSO. Cho phép quản lý hệ thống thêm, bớt máy in; thực hiện duyệt các tài liệu được yêu cầu in.  \par
\item Printer Staff Interface: Chế độ dành cho nhân viên gác máy in. Cho phép nhân viên gác máy có thể thấy được lịch sử gồm các tài liệu đã qua kiểm duyệt cần được in qua đó để thực hiện việc in ấn cho sinh viên.  \par

\end{itemize}
\item Interface Controller: Là lớp chịu trách nhiệm điều phối các tương tác giữa các giao diện người dùng (UI) và các lớp logic kinh doanh (Business logic) trong ứng dụng.\par


\item Business Layer (Lớp logic kinh doanh): Lớp này chứa logic của ứng dụng. \par   

\begin{itemize}
\item Thực hiện việc in ấn tài liệu cho sinh viên. \par
\item Xem lại lịch sử các tài liệu.\par
\item Quản lý hệ thống.\par
\item Thực hiện thanh toán.\par

\end{itemize}
\item Database Layer (Lớp dữ liệu): Lớp này quản lý việc truy cập và tương tác với cơ sở dữ liệu.\par    
\begin{itemize}
\item Printer Database: Lưu trữ danh sách các máy in có trong khuôn viên trường. 
\item History Database: Lưu trữ lịch sử các tài liệu yêu cầu/được in của hệ thống. \par

\end{itemize}

\item Server Manager: Điều phối tương tác giữa Database và các tác vụ ở lớp Business.\par
\item External services/APIs (Các dịch vụ nằm ngoài phạm vi của ứng dụng chính): Là các giao diện được cung cấp bởi các bên thứ ba. Hệ thống liên kết với các External services thông qua:\par    
\begin{itemize}
\item BK\textunderscore Pay: Thông qua chức năng Payment của lớp Business, chúng ta sẽ tạm thời rời hệ thống hiện tại để đi tới giao diện của BK\textunderscore Pay. 
\item HCMUT\textunderscore SSO: Thông qua Authetication của Interface controller, hệ thống sẽ đi tới giao diện của HCMUT\textunderscore SSO. \par

\end{itemize}


\end{itemize}


\subsubsection{Deployment}
\begin{figure}[H]
    \centering
\includegraphics[scale=0.35]{Images/architecture-diagram/deploy.png}
    \vspace{0.1cm}
    \caption{Deployment diagram }
\end{figure}
\begin{itemize}
\item Sinh viên và SPSO kết nối với application server thông qua trình duyệt web sử dụng phương thức TCP/IP. \par
\item Application server sẽ cung cấp chức năng đăng nhập cho mọi đối tượng.  \par
\item Sau khi đăng nhập, Application server sẽ sử dụng dịch vụ xác thực do HCMUT SSO cung cấp bằng cách gọi API theo phương thức HTTP được cung cấp từ HCMUT SSO, kết quả trả về sẽ xác định đối tượng đăng nhập là  sinh viên hay SPSO để cung cấp giao diện và dịch vụ tương ứng.\par    


\item Application server cung cấp giao diện và dịch vụ in ấn, xem lịch sử và thanh toán cho sinh viên sử dụng, quản lí cấu hình hệ thống, quản lí máy in, xem toàn bộ lịch sử cho SPSO. \par   


\item Database system sử dụng MongoDB atlas để lưu dữ liệu về máy in,số trang của mỗi sinh viên cũng như lịch sử in ấn. Application server giao tiếp với database system qua internet sử dụng MongoDB Wire Protocol và định tuyến bằng TCP/IP.  \par
\item Với các dịch vụ bên ngoài khác, được thiết kế để sử dụng trong artificial external service/ apis như trên hình. Sử dụng kết nối theo chuẩn được cung cấp từ nhà cung cấp dịch vụ. Ví dụ, đối với BKPay để thanh toán cho sinh viên.Application server có thể gửi yêu cầu HTTP đến hệ thống BKPay. Hệ thống BKPay sẽ trả lại URL thanh toán. Application server sau đó có thể chuyển hướng người dùng đến URL thanh toán để hoàn tất thanh toán.Để nhận thông báo khi thanh toán mua trang in được hoàn tất,Application server có thể đăng ký nhận webhooks được cung cấp bởi hệ thống BKPay. Khi thanh toán được hoàn tất, hệ thống BKPay sẽ gửi thông báo qua webhook. Từ đó có thể cập nhật số dư trang in của số trang cho phù hợp và cập nhật lại trong cơ sở dữ liệu.\par

\end{itemize}




\newpage
\subsection{Component Diagram}
\subsubsection{In tài liệu}
\begin{figure}[H]
    \centering
    \includegraphics[scale=0.38]{Images/component-diagram/printing.png}
    \vspace{0.1cm}
    \caption{Component diagram của nhánh chức năng in tài liệu}
\end{figure}
\begin{itemize}
    \item Hệ thống sẽ có 3 component lớn
    \begin{itemize}
        \item PrintPage component:
        \begin{itemize}
            \item Bao gồm Student Interface component
            \item Các component trong Student Interface đảm nhận nhiệm vụ như đã mô tả trong task 2.3
            \item Các component này yêu cầu interface từ PrintController để thực hiện lấy thông tin cần thiết.
        \end{itemize}
        \item PrintController component:
        \begin{itemize}
            \item Chứa các component Printer Controller, Properties và File Controller.
            \item Các component này đảm nhận các nhiệm vụ như đã mô tả trong task 2.3
            \item Properties và File Controller sẽ cung cấp interface cho Printer Controller để thực hiện việc kiểm tra định dạng file và các thuộc tính in.
            \item Ngoài ra PrintController sẽ cũng cấp các interface cho PrintPage để lấy thông tin và yêu cầu interface từ PrintModel
        \end{itemize}
        \item PrintModel component:
        \begin{itemize}
            \item Chứa các component Printer và File
            \item Các component này sẽ đảm nhận các nhiệm vụ như đã mô tả trong task 2.3
            \item PrintModel sẽ cung cấp các interface cho PrintController để lấy thông tin danh sách máy in và các file tài liệu được tải lên
        \end{itemize}
    \end{itemize}
    \item Ngoài ra còn có các component:
    \begin{itemize}
        \item BKPay component:
        \begin{itemize}
            \item Được dùng để thực hiện các thanh toán
            \item Cung cấp interface để PrintPage thực hiện thao tác với dữ liệu về payment
        \end{itemize}
        \item Database component:
        \begin{itemize}
            \item Thao tác trực tiếp với cơ sở dữ liệu
            \item Cung cấp các interface để PrintModel thực hiện truy xuất trực tiếp thông tin trên cơ sở dữ liệu
        \end{itemize}
    \end{itemize}
\end{itemize}

\subsubsection{Quản lý máy in}
\begin{figure}[H]
    \centering
    \includegraphics[width = 1\textwidth]{Images/component-diagram/component_diagram_manage_printer.png}
    \newline
    \newline
    \caption{Component diagram của nhánh chức năng quản lý máy in}
    \label{fig:enter-label}
\end{figure}

\textbf{\textit{Mô tả:}}
\begin{itemize}
    \item Trong giao diện của nhân viên dịch vụ in ấn (SPSO), dữ liệu về danh sách các máy in (\textit{Printer List}) có trong \textit{Printer Database} được tổng hợp bởi khối \textit{Get Printer Listing Info} và xuất ra giao diện thông qua khối \textit{Display Printer Listing}, tương tự dữ liệu chi tiết của một máy in cụ thể cũng được tổng hợp và xuất ra giao diện \textit{Display Printer Details}
    \item Thao tác bật/tắt máy in được xử lý thông qua khối \textit{Change Printer State}. Khối này nhận thông tin về trạng thái (bật/tắt) của máy in được SPSO chọn từ giao diện được xuất bởi khối \textit{Display Enable/Disable Printers}, sau đó cập nhật trạng thái đó vào máy in được chọn trong \textit{Printer Database}

    \item SPSO thao tác việc thêm máy in mới vào hệ thống thông qua giao diện được xuất bởi khối \textit{Display Add Printer}, cung cấp thông tin của máy in mới muốn thêm vào để khối \textit{Add new printer} xử lý. Khối \textit{Add new printer} sau khi nhận được thông tin thì tiến hành cập nhật máy in mới vào \textit{Printer List} trong component \textit{Printer Database}
\end{itemize}

\subsubsection{Tải tài liệu}
\begin{figure}[H]
    \centering
    \includegraphics[width = 1\textwidth]{Images/component-diagram/Uploadfile.png}
    \newline
    \newline
    \caption{Component diagram của nhánh chức năng tải tài liệu}
    \label{fig:enter-label}
\end{figure}
\begin{itemize}
    \item Hệ thống sẽ có 3 component lớn
    \begin{itemize}
        \item PrintPage component:
        \begin{itemize}
            \item Bao gồm Student Interface component
            \item Các component trong Student Interface đảm nhận nhiệm vụ như đã mô tả trong task 2.3
            \item Các component này yêu cầu interface từ PrintController để thực hiện lấy thông tin cần thiết.
        \end{itemize}
        \item PrintController component:
        \begin{itemize}
            \item Chứa component File Controller.
            \item Component này đảm nhận các nhiệm vụ như đã mô tả trong task 2.3
            \item File Controller sẽ cung cấp interface cho Printer Controller để thực hiện việc kiểm tra định dạng file.
            \item Ngoài ra PrintController sẽ cũng cấp các interface cho PrintPage để lấy thông tin và yêu cầu interface từ PrintModel
        \end{itemize}
        \item PrintModel component:
        \begin{itemize}
            \item Chứa component File
            \item Các component này sẽ đảm nhận các nhiệm vụ như đã mô tả trong task 2.3
            \item PrintModel sẽ cung cấp các interface cho PrintController để lấy thông tin danh sách các file tài liệu được tải lên
        \end{itemize}
    \end{itemize}
    \item Ngoài ra còn có Database component:
        \begin{itemize}
            \item Thao tác trực tiếp với cơ sở dữ liệu
            \item Cung cấp các interface để PrintModel thực hiện truy xuất trực tiếp thông tin trên cơ sở dữ liệu
        \end{itemize}
    \end{itemize}
